\documentclass[10pt,letterpaper]{article}

\usepackage[spanish]{babel}
\usepackage{hyperref}


\usepackage{listings}
\usepackage{color}

\definecolor{mygreen}{rgb}{0,0.6,0}
\definecolor{mygray}{rgb}{0.5,0.5,0.5}
\definecolor{mymauve}{rgb}{0.58,0,0.82}

\lstset{ 
	backgroundcolor=\color{white},   % choose the background color; you must add \usepackage{color} or \usepackage{xcolor}; should come as last argument
	basicstyle=\footnotesize,        % the size of the fonts that are used for the code
	breakatwhitespace=false,         % sets if automatic breaks should only happen at whitespace
	breaklines=true,                 % sets automatic line breaking
	captionpos=b,                    % sets the caption-position to bottom
	commentstyle=\color{mygreen},    % comment style
	deletekeywords={...},            % if you want to delete keywords from the given language
	escapeinside={\%*}{*)},          % if you want to add LaTeX within your code
	extendedchars=true,              % lets you use non-ASCII characters; for 8-bits encodings only, does not work with UTF-8
	firstnumber=1000,                % start line enumeration with line 1000
	frame=single,	                   % adds a frame around the code
	keepspaces=true,                 % keeps spaces in text, useful for keeping indentation of code (possibly needs columns=flexible)
	keywordstyle=\color{white},       % keyword style
	language=Octave,                 % the language of the code
	morekeywords={*,...},            % if you want to add more keywords to the set
	numbers=left,                    % where to put the line-numbers; possible values are (none, left, right)
	numbersep=5pt,                   % how far the line-numbers are from the code
	numberstyle=\tiny\color{mygray}, % the style that is used for the line-numbers
	rulecolor=\color{black},         % if not set, the frame-color may be changed on line-breaks within not-black text (e.g. comments (green here))
	showspaces=false,                % show spaces everywhere adding particular underscores; it overrides 'showstringspaces'
	showstringspaces=false,          % underline spaces within strings only
	showtabs=false,                  % show tabs within strings adding particular underscores
	stepnumber=2,                    % the step between two line-numbers. If it's 1, each line will be numbered
	stringstyle=\color{mymauve},     % string literal style
	tabsize=2,	                   % sets default tabsize to 2 spaces
	title=\lstname                   % show the filename of files included with \lstinputlisting; also try caption instead of title
}


\title{Tutorial de Programación en Bash}
\author{Cristian Segura}

\begin{document}

\maketitle
\tableofcontents

\section{Introducción}
\subsection{¿Qué es Bash?}
Bash es una interfaz de línea de comandos utilizada inicialmente (o Shell)

\subsection{Conseguir una Computadora con Linux}
En la actualidad (2023), hay varias formas de lograr acceder a una computadora con Linux para 
poder comenzar a utilizar Bash.
Algunas de las más populares son:

\begin{itemize}
	\item \textit{Crear una Máquina Virtual}. Esto consiste en crear un máquina virtual que se ejecute dentro de su sistema operativo y ahí instalar alguna distribucion de Linux. Para ello necesita instalar previamente software de virtualización de escritorio como VirtualBox (Windows, MacOS), VMWare Workstation (Windows), Parallels (MacOS) o HyperV (Windows). Puede encontrar un tutorial de como realizar la instalación en la siguiente página \href{http://url.com/instalar-ubuntu-22-virtualbox-windows}{web}
	
	\item \textit{Instalar Bash en su Sistema} (solo para MacOS). .Puede encontrar un tutorial de como realizar la instalación en la siguiente página \href{http://url.com/bash-en-macos}{web}
	
	\item \textit{Instalar WSL2 en su sistema} (solo para Windows). .Puede encontrar un tutorial de como realizar la instalación en la siguiente página \href{http://url.com/instalar-wsl2}{web}
	
	\item \textit{Comprar una placa Raspberry Pi}. La popular placa \href{https://www.raspberrypi.org/}{Raspberry Pi} utiliza una versión de Linux llamada \href{https://www.raspberrypi.com/software/operating-systems/}{Raspbian} y ya viene con Bash instalado.
\end{itemize}

Adicionalmente puede crear una Máquina Virtual con Linux en algún proveedor Cloud como \href{https://www.digitalocean.com/}{Digital Ocean}, \href{https://www.linode.com/}{Linode}, \href{https://aws.amazon.com}{Amazon AWS} o \href{https://azure.microsoft.com/}{Azure}. Sin embargo, para crear estas máquinas se requiere pago y el ingreso de una tarjeta de crédito.



\section{El prompt de Bash}

Al entrar a un terminal ejecutando Bash aparece una lína similar a la que se muestra:

\begin{lstlisting}[language=Bash]
csegura@csegura-VirtualBox:~$
\end{lstlisting}
Al imprimir esta línea, llamada "línea de comando" (en inglés \textit{command line}), el sistema nos está indicando que se encuentra a la espera que escribamos algún comando. Pero antes de escribir nada es interesante entender el significado de la línea que estamos viendo.
\\
En la línea de comando se puede observar, en la mayoría de los casos, la siguiente sintaxis:

\begin{lstlisting}[language=Bash]
<user>@<hostname>:~$ _
\end{lstlisting}

Los campos que aparecen son:

\begin{itemize}

\item  El campo \verb+<user>+ indica el nombre del usuario actual.

\item  El símbolo de arroba \verb+@+ (llamado \verb+at+ en inglés) significa \textbf{en} o \textbf{dentro de}.

\item  El campo \verb+<hostname>+ indica nombre del host (o *hostname* en la jerga de redes de computadoras) es el nombre que se le asigna a la computadora que está corriendo Bash. En general este nombre puede ser cualquier tira de caracteres. Nombres típicos son: \verb+localhost+, \verb+ubuntu+, \verb+lnxsrv+, etc.. (básicamente cualquier nombre se le quiera dar a la máquina). Es interesante notar que \verb+<user>@<hostname>+ le recuerda que se encuentra en la máquina llamada \verb+<hostname>+ utilizando el usuario \verb+<user>+.

\item  El símbolo virgulilla \verb+~+ indica que nos encontramos en el directorio \verb+HOME+ del usuario \verb+<user>+. Veremos más en detalle este punto cuando revisemos los comandos para navegar por el sistema de archivos.

\item El símbolo virgulilla \verb+\$+ : este símbolo que se termina el prompt y que todo lo que está a la derecha es un comando que se le estará ingresando a Bash.

\end{itemize}


Por ejemplo cuando abro una terminal con Bash aparece el siguiente prompt:

\begin{lstlisting}[language=Bash]
csegura@csegura-VirtualBox:~$
\end{lstlisting}


	
\end{document}